\pagenumbering{arabic}
\chapter*{Introduction General}
\addcontentsline{toc}{chapter}{Introduction générale} % to include the introduction to the table of content
\markboth{Introduction générale}
L’évolution exponentielle des TIC a fait naître une nouvelle génération d’outils qui on pu changer plusieurs aspects de notre vie quotidienne. Ces outils ont apporté de l’innovation dans plusieurs secteurs et, désormais, ils jouent un rôle important dans la
compétitivité de toute entreprise ainsi que dans l’administration des services publics.
\\
\\
Traditionnellement, et même de nos jours une formation en intra entreprise implique
systématiquement la présence d’un formateur, la mise à disposition de locaux, nécessite des
éventuels déplacements pour les collaborateurs, un bouleversement dans l’organisation du
travail et une certaine rigidité en termes d’horaires. Mais la révolution digitale et la digitalisation de l’information qui l’a accompagnée n’a pas complètement fait disparaître nos
habitudes « physiques » et nombreux d’entre nous sont encore attachés à des façons de faire plus classiques, aussi bien dans la sphère privée que professionnelle.
\\
\\C’est dans ce contexte que la société Inno-Think a décidé de la mise en place d’une plateforme de formation continue.
En effet, notre travail consiste à développer une plateforme qui répond à ses besoins et qui sera intitulé « CodeNest ».
\\
\\Le présent rapport présentera les différentes
étapes de la réalisation de ce projet et qui sera exposé via les différents chapitres.
\\Le chapitre 1 « Présentation du cadre génerale » est consacré à la présentation de l’établissement d’accueil, l’objectif du travail à réaliser ainsi qu’à la méthodologie de gestion de projet adopté.
Le chapitre 2 «  Mise en œuvre de projet » va être une base pour l’analyse la spécification des besoins de notre projet d’une part et d'autre part pour choix technologique et la conception.
\\Les chapitres 3,4 et 5 ont pour objectif de montrer les différentes étapes suivies pour la mise en œuvre respectivement du Sprint 1,2 et 3 .
La dernière partie de ce rapport est consacrée à la conclusion et les perspectives du travail.
\\
\\Ce chapitre sera consacré à la définition de la méthode de gestion de projet, leur comparaison et le choix de la méthode appropriée à cet projet. 